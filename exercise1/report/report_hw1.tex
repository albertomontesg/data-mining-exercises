\documentclass{ethz_report}
\usepackage{listings}
\usepackage{color}

\definecolor{codegreen}{rgb}{0,0.6,0}
\definecolor{codegray}{rgb}{0.5,0.5,0.5}
\definecolor{codepurple}{rgb}{0.58,0,0.82}
\definecolor{backcolour}{rgb}{1,1,1}

\lstdefinestyle{mystyle}{
    backgroundcolor=\color{backcolour},
    commentstyle=\color{codegreen},
    keywordstyle=\color{magenta},
    numberstyle=\tiny\color{codegray},
    stringstyle=\color{codepurple},
    basicstyle=\ttfamily,
    breakatwhitespace=false,
    breaklines=true,
    captionpos=b,
    keepspaces=true,
    numbers=left,
    numbersep=5pt,
    showspaces=false,
    showstringspaces=false,
    showtabs=false,
    tabsize=4,
    frame=lines
}
\lstset{style=mystyle}

\title{Assigment 1 - MapReduce}
\subject{Data Mining: Learning from Large Datasets}
\author{Alberto Montes}
\email{malberto@student.ethz.ch}
\date{\today}

\begin{document}
\maketitle

\section{Problem 1: Approximation of the English Dictionary}

For this problem, the Map function consist on detect all the words (with a Regex expression
extracting them) as first step. At each word found, it returns the word as key and a one
indicating one found of the word. With this approach, the counting of all the appearance of the
words are done at the reducer function.

\lstinputlisting[language=Python, caption=mapper1.py]{../code/mapper1.py}

As specified, the Reducer opperation keeps a counter of how many times a word has appeared, and when
a new word appeards, it prints the total amount of times the previous word has appeared.

\lstinputlisting[language=Python, caption=reducer1.py]{../code/reducer1.py}


\section{Problem 2: A Basic English Dictionary}

For this second problem, as the Map function can be used the same as in Problem 1. The challange is
in modifying the Reducer function.

\lstinputlisting[language=Python, caption=mapper2.py]{../code/mapper2.py}

The Reducer opperation keeps track of the current character of the dictionary and stores the
words and counts in a heap until the maximum capacity per character which specified as 30 per
letter. To store the word in the heap, the count of the word must fullfill the requirements set by
the problem statement.
The count of the words are kept track as in the first problem. Finally when the first letter changes
the heap is returned and as it has been stored, it is return sorted alphabeticcaly.

\lstinputlisting[language=Python, caption=reducer2.py]{../code/reducer2.py}


\end{document}
